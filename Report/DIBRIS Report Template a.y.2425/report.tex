\documentclass{article}

\usepackage{dibrisunige-report}
\usepackage{codespace}

% Sub-preambles
% https://github.com/MartinScharrer/standalone

% Encodings
\usepackage{amsmath,amssymb,gensymb,textcomp}

% Better tables
% Wide tables go to https://tex.stackexchange.com/q/332902
\usepackage{array,multicol,multirow,siunitx,tabularx}

% Better enum
\usepackage{enumitem}

% Graphics
\usepackage{caption,float}

% Allow setting >max< width of figure
% 'export' allows adjustbox keys in \includegraphics
\usepackage[export]{adjustbox}

% For demonstration purposes, remove in production
\usepackage{mwe}

% Configurations
\newcounter{memberrowno}
\setcounter{memberrowno}{0}

\ocoursename{Course name}
\oreporttype{Report}
\title{Project title}
\oadvisor{Advisor/s}
\reportlayout%

% Custom commands
\newcommand*\mean[1]{\bar{#1}}

% Title and Metadata
\title{[Your Project Title]}
\author{[Your Name(s)]}
\date{\today}

\begin{document}

% Cover Page
\coverpage%

\section*{Member list \& Workload}
\begin{center}
  \begin{tabular}{>{\stepcounter{memberrowno}\thememberrowno}llcc}
    \toprule
    \multicolumn{1}{c}{\textbf{No.}} & \textbf{Full name} & \textbf{Student ID} & \textbf{Percentage of work} \\
    \midrule
                                     & h                  & xxxxxxx             & 100\%                       \\
                                     & h                  & xxxxxxx             & 100\%                       \\
    \bottomrule
  \end{tabular}
\end{center}

\newpage
% Table of Contents
\tableofcontents
\newpage


% Abstract
%\section{Abstract (max 500 chars)}
%This is how you normally work with \LaTeX, but you can also split a project into smaller files for easier management.
%\\
%Here a few lines to describe the goal of the project, the  solution/s adopted and (in case) the difficulties faced and novelties proposed. 
%\\
%To import other files, you can use \mintinline{latex}{\input{}} or \mintinline{latex}{\include{}}.
%There differences can be found at \url{https://tex.stackexchange.com/a/250}, but in short
%
%\[\mintinline{latex}{\include{filename}} = \mintinline{latex}{\clearpage \input{filename} \clearpage}\]

% Concept Overview
\section{Concept Overview (max 1000 chars)}
\begin{itemize}
    \item \textbf{Brief Description:} [A concise explanation of what your project is and what it does]
    \item \textbf{Objectives:} [The main goals you aim to achieve with your project]
    \item \textbf{Key Feature:} [The most distinctive or innovative aspect of your project]
\end{itemize}

% Target Audience
\section{Target Audience (a few lines)}
\begin{itemize}
    \item \textbf{Primary Users:} [Describe your primary target audience, e.g., "Remote workers"]
    \item \textbf{Secondary Users:} [Describe other potential users, if any]
\end{itemize}

% Example of Use Case
\section{Example of Use Case (max 1000 char)}
\textbf{Description:} [A sequence of actions performed by a user to achieve a specific goal within the application]

% Technical Information
\section{Technical Information (a short but precise list}
\begin{itemize}
    \item \textbf{Technologies \& Software:} [The specific technologies you'll be using, e.g., "Unreal Engine version XX.YY, ChatGPT, Convai, MetaHuman, etc."]
    \item \textbf{Platforms:} [Specify the platforms on which the application will be available, e.g., "Oculus, iOS, Android, Web"]
    \item \textbf{Programming Languages:} [List the programming languages you will use, e.g., "Blueprint, C++, Python, Swift, Kotlin, JavaScript"]
    \item \textbf{Frameworks:} [Specify the frameworks you will use, e.g., "React Native, Flutter"]
    \item \textbf{Database:} [Specify the type of database, if necessary, e.g., "MongoDB"]
    \item \textbf{Cloud:} [Specify the cloud platform you will use, if necessary, e.g., "OneDrive, AWS, Google Cloud"]
\end{itemize}

% Development Plan
\section{Development Plan (a few lines with indication of expected date of presentation of the demo)}
\begin{itemize}
    \item \textbf{Team and Roles:} [Who is on the team and what are their responsibilities]
    \item \textbf{Timeline:} [The major milestones in your development process (concept, prototype, beta, launch)]
\end{itemize}


% State or Art
\section{State of Art (max 1 page)}
Here the state of art  related of topic of your project. \\
E.g. In recent years, thanks to technological progress, more and more research fields have borrowed from scientific circles the use of collecting data in digital format. This is the case of the XXX field, which began to use open data to collect and share information... \\

The State of Art should briefly place the study in a broad context and highlight why it is important. It should define the purpose of the work and its significance. The current state of the research field should be reviewed carefully and key publications cited using BibTex. Please highlight controversial and diverging hypotheses when necessary. Finally, briefly mention the main aim of the work and highlight the principal conclusions. As far as possible, please keep the introduction comprehensible to scientists outside your particular field of research. Citing a journal paper \cite{knuth:1984}. Now citing a book reference \cite{texbook, latex:companion} or other reference types like conferences \cite{lesk:1977}.\\

Optional: insert a subsection in case of a relevant subtopic (e.g. SLAm, teleportation, etc.)\\


\subsection{Sample: Open Data and Linked Open Data}
Open data are data usable by everybody, that is, they are a type of data that is made available to anyone who wants to access, reuse or redistribute it, on the condition that you know the origin and the author. This promotes the transparency and reliability of data applications for all types of scope, from administrative sector to the cultural arts environment. The availability of data is also a great advantage for shared knowledge made accessible via the network and the Internet, for example.
\\

Linked Open Data (LOD) is a way of publishing structured data that allows metadata to be connected and enriched, so that different representations of the same content can be found, and links made between related resources. All LOD datasets can be explored and queried through the SPARQL API.


\subsection{Sample: Cultural heritage ontologies}
Thanks to the application of Linked Open Data \cite{LOD} in the field of the conservation of cultural heritage, we were able to study projects that have ontologies uploaded on the Web.   
\\
Here some of the ontologies that we analyzed and which we have took inspiration from: 
\begin{enumerate}
    \item \href{https://dati.beniculturali.it/cultural-ON/ITA.html}{Ontologia dei Luoghi della Cultura e degli Eventi Culturali}
    \item \href{http://www.san.beniculturali.it/web/san/ontologia-san-lod}{Sistema archivistico nazionale (SAN)}
    \item \href{https://www.icar.beniculturali.it/attivita-e-progetti/progetti-icar-1/linked-open-data}{Istituto Centrale per gli Archivi - ICAR}
\end{enumerate}

% other chapters for Final Report
%\include{chapters/tools.tex}
%\include{chapters/description.tex}
%\include{chapters/results.tex}
%\include{chapters/conclusions.tex}

% some example of chapters
%\include{chapters/example/bettertables.tex}
%\include{chapters/example/betterenum.tex}
%\include{chapters/example/codespace.tex}
%\include{chapters/example/figwidth.tex}


\bibliographystyle{plain}
\bibliography{refs}
\nocite{*}
\end{document}
